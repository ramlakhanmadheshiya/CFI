\documentclass[conference]{IEEEtran}
\IEEEoverridecommandlockouts
% The preceding line is only needed to identify funding in the first footnote. If that is unneeded, please comment it out.
\usepackage{cite}
\usepackage{amsmath,amssymb,amsfonts}
\usepackage{algorithmic}
\usepackage{graphicx}
\usepackage{textcomp}
\usepackage{xcolor}
\def\BibTeX{{\rm B\kern-.05em{\sc i\kern-.025em b}\kern-.08em
    T\kern-.1667em\lower.7ex\hbox{E}\kern-.125emX}}
\begin{document}

\title{Windows CVE Analysis: Vulnerabilities, Exploits, and Defenses\\
}

\author{\IEEEauthorblockN{1\textsuperscript{st} Author Name}
\IEEEauthorblockA{\textit{Department of Computer Science} \\
\textit{University Name}\\
City, Country \\
email@university.edu}
\and
\IEEEauthorblockN{2\textsuperscript{nd} Author Name}
\IEEEauthorblockA{\textit{Department of Computer Science} \\
\textit{University Name}\\
City, Country \\
email@university.edu}
\and
\IEEEauthorblockN{3\textsuperscript{rd} Author Name}
\IEEEauthorblockA{\textit{Department of Computer Science} \\
\textit{University Name}\\
City, Country \\
email@university.edu}
}

\maketitle

\begin{abstract}
The escalating complexity of Windows vulnerabilities and advanced persistent threats necessitates comprehensive understanding of contemporary attack vectors and defense mechanisms. Through systematic analysis of 28 research papers published between 2021-2025, this review examines critical developments in Windows CVE exploitation, malware evolution, and defensive strategies. Novel vulnerability discovery tools (JERRY, LinkZard, GLEIPNIR) have identified 394+ previously unknown security flaws, while traditional defenses face increasing sophistication in evasion techniques. This paper synthesizes key trends in vulnerability management, establishes research gaps in adaptive defense mechanisms, and provides strategic recommendations for risk-based vulnerability prioritization frameworks.
\end{abstract}

\begin{IEEEkeywords}
Windows vulnerabilities, CVE analysis, exploitation patterns, vulnerability management, ransomware, memory corruption, ASLR, defense strategies, threat intelligence, MITRE ATT\&CK, zero-day vulnerabilities, risk-based prioritization
\end{IEEEkeywords}

\section{Introduction}

The security landscape of the Windows operating system presents a complex and evolving challenge that extends far beyond simple vulnerability enumeration. With approximately 70\% of the global desktop market share \cite{burke2024}, Windows remains the ``most targeted operating system'' for cybercriminals, nation-state actors, and opportunistic threat actors alike. The relentless pace of vulnerability discovery, coupled with the sophistication of modern exploitation techniques, has fundamentally transformed the discipline of vulnerability analysis from a reactive cataloging exercise into a proactive, intelligence-driven science.

The Common Vulnerabilities and Exposures (CVE) system, while serving as the de facto standard for vulnerability identification, presents both opportunities and challenges for security professionals. Recent research demonstrates that the sheer volume of disclosed vulnerabilities has reached a critical threshold where traditional patch management strategies have become operationally infeasible \cite{shimizu2025, onyagu2025}. Microsoft's monthly ``Patch Tuesday'' releases routinely include dozens of vulnerabilities, many classified as ``Critical'' or ``High'' severity, creating a prioritization dilemma for enterprise security teams with limited resources and operational constraints.

This paper addresses a critical gap in the current literature by providing a comprehensive, multi-dimensional analysis of Windows vulnerability research that synthesizes three historically separate domains: (1) the quantitative and predictive analysis of vulnerability data, examining temporal trends and evolutionary patterns; (2) the technical investigation of exploitation methodologies, ranging from foundational memory corruption techniques to novel, non-traditional attack surfaces; and (3) the systematic evaluation of defensive strategies, encompassing both software-based hardening and hardware-rooted security architectures.

\subsection{Evolution and Challenges}

Vulnerability research has transformed from static CVE enumeration to predictive analytics. Saklani et al. \cite{saklani2023} analyzed 77,202 records showing traditional buffer errors (CWE-119) decreasing while logical flaws increase. Williams et al. \cite{williams2020} demonstrate CVEs exist within evolutionary progression, enabling predictive modeling. Modern exploitation evolved from stack overflows to DEP bypass via ROP, to ASLR circumvention \cite{zumbo2024}. Critically, Xiang et al. \cite{xiang2025} and Yu et al. \cite{yu2024} discovered 394 combined zero-day vulnerabilities in overlooked logical flaw classes.

Research exposes defense limitations: Binosi et al. \cite{binosi2024} reveal Windows 11's ASLR provides only 16-18 bits entropy for critical components; Dhokley et al. \cite{dhokley2025} show scanners lag significantly behind CVE publication. Most significantly, Shimizu \& Hashimoto \cite{shimizu2025} demonstrate CVSS-only strategies create ``vulnerability overload,'' while KEV/EPSS integration achieves 18-fold efficiency improvement with 95\% workload reduction maintaining 85.6\% coverage of exploited vulnerabilities.

\subsection{Research Objectives}

This review synthesizes 28 research papers (2021-2025) to: (1) identify temporal vulnerability trends and evolutionary patterns, (2) catalog exploitation methodologies across memory corruption, logical flaws, and novel attack surfaces (Secure Kernel VTL1, IPC clients), (3) evaluate defense mechanism efficacy and limitations, (4) analyze risk-based vulnerability management frameworks integrating threat intelligence. Section II presents literature review; Section III describes methodology; Section IV provides discussion and analysis; Section V concludes with recommendations. 

\section{Literature Review}

\begin{figure}[!t]
\centering
\includegraphics[width=3.5in]{diagram 2.png}
\caption{Contemporary Vulnerability Landscape}
\label{fig:landscape}
\end{figure}

\subsection{Vulnerability Trends and Predictive Modeling}

Quantitative vulnerability analysis reveals critical evolutionary patterns. Softić \& Vejzović \cite{softić2022} analyzed Windows 10, macOS, and Ubuntu vulnerabilities (2015-2021), finding Windows 10 exhibited increasing trends with code execution as the dominant type. Saklani et al. \cite{saklani2023} examined 77,202 NVD records (2016-2021), revealing buffer errors (CWE-119) decreasing while logical flaws (CWE-79, CWE-269, CWE-863) increased. Critical vulnerabilities decreased; low-severity increased. Microsoft products showed increasing vulnerability trends after 2017.

Williams et al. \cite{williams2020} introduced evolutionary modeling using Topically Supervised Evolution Model (TSEM), demonstrating CVEs exist in evolutionary narratives. Their framework traced how CVE-2015-6128 evolved from CVE-2014-0318, enabling predictive estimation of future vulnerability groups through diffusion-based storytelling.

\begin{table}[htbp]
\caption{Vulnerability Type Evolution (2016-2021)}
\begin{center}
\begin{tabular}{|l|c|c|}
\hline
\textbf{Vulnerability Type} & \textbf{CWE} & \textbf{Trend} \\
\hline
Buffer Errors & CWE-119 & Decreasing \\
\hline
Access Control & CWE-264 & Decreasing \\
\hline
Cross-Site Scripting & CWE-79 & Increasing \\
\hline
Privilege Management & CWE-269 & Increasing \\
\hline
Incorrect Authorization & CWE-863 & Increasing \\
\hline
Code Execution & - & Dominant \\
\hline
\end{tabular}
\label{tab:vulntrends}
\end{center}
\end{table}

\subsection{Exploitation Patterns and Novel Attack Surfaces}

The Windows exploitation landscape demonstrates continuous evolution through increasingly sophisticated techniques. Zumbo \cite{zumbo2024} documents the historical arms race: stack overflows countered by DEP, leading to ROP attacks, subsequently mitigated by ASLR. However, Binosi et al. \cite{binosi2024} reveal Windows 11's ASLR provides only 16-18 bits entropy for executables/libraries versus 23-31 bits for runtime objects, enabling practical bruteforce attacks. Lee et al. \cite{lee2021} introduced EXPRACE, exploiting kernel races via IPI manipulation to deterministically trigger race conditions. Pfau \& Kochberger \cite{pfau2024} demonstrate CFG protects indirect calls but not direct return overwrites, leaving classic stack overflows unmitigated.

\textbf{Novel Vulnerability Classes:} Recent research identifies previously overlooked logical flaw categories. Yu et al. \cite{yu2024} discovered 339 File Hijacking Vulnerabilities using JERRY tool, exploiting insecure file system trust (e.g., CVE-2022-24765 in Git). Xiang et al. \cite{xiang2025} found 55 Link Following Vulnerabilities with LinkZard, exploiting symbolic link validation failures. Gu et al. \cite{gu2025} identified 25 server-induced IPC client vulnerabilities using GLEIPNIR, inverting the traditional threat model where compromised low-privilege servers exploit high-privilege clients (\$36K bounty, 14 CVEs). Jagt \cite{jagt2025} discovered VTL1 Secure Kernel TOCTOU vulnerabilities through n-day analysis, proving even the most hardened components contain exploitable flaws. Combined, these tools revealed 394+ zero-days immune to all memory protections (DEP, ASLR, CFG).

\begin{figure}[!t]
\centering
\includegraphics[width=3.5in]{diagram 4.png}
\caption{Automated Vulnerability Discovery}
\label{fig:automated_discovery}
\end{figure}

\textbf{Contemporary Threats:} Ojo \cite{ojo2025} documents ransomware evolution to double extortion and RaaS platforms. Syifa \& Salman \cite{syifa2025} validated SMB/RDP weaknesses via Caldera emulation, identifying misconfigurations as persistent attack vectors. Kasaza \cite{kasaza2024} demonstrated USB HID attacks using Raspberry Pi Zero W, achieving full system compromise in seconds through automated keystroke injection. Shinde \& Khobragade \cite{shinde2022} model malware behavior through Knowledge Graphs mapping DLL imports and API calls, enabling classification of entire threat families.

\subsection{Defense Mechanisms and Limitations}

\subsubsection{Malware Detection Technologies and Challenges}

The contemporary defensive landscape has evolved to incorporate sophisticated machine learning and deep learning techniques, yet faces persistent challenges that limit effectiveness. Maniriho et al. \cite{maniriho2023} provide a comprehensive systematic literature review of Windows malware detection techniques published between 2009 and 2022, offering detailed analysis of the defensive state-of-the-art.

Detection approaches are categorized into three main methodologies: (1) \textit{Static Analysis}, which detects malware without execution by matching static signatures (e.g., hash values, strings) or analyzing PE file headers, opcodes, and API call imports; (2) \textit{Dynamic Analysis (Behaviour-based)}, which detects malware by executing it in isolated sandbox environments and monitoring behavior such as running processes, API calls, network behaviors, and loaded dynamic link libraries; and (3) \textit{Hybrid Analysis}, combining both static and dynamic features to achieve more robust detection while mitigating weaknesses of each individual method.

The review confirms widespread adoption of machine learning and deep learning algorithms for modern malware detection. The most popular ML algorithms identified are Random Forest (RF) and Support Vector Machines (SVM), while the most prevalent DL models are Convolutional Neural Networks (CNN), often used for image-based malware classification by converting binaries to visual representations, and Recurrent Neural Networks (RNNs), including Long Short-Term Memory (LSTM), which are well-suited for analyzing sequential data like API call sequences.


Critically, the research identifies fundamental issues impeding defensive effectiveness: \textit{Experimental Biases} (temporal/spatial bias inflate results), \textit{Concept Drift} (models deteriorate as malware evolves), \textit{Adversarial Attacks} (adversarial ML fools classifiers), and \textit{Evasion Mechanisms} (fileless malware via PowerShell/WMI bypasses static analysis).

\subsection{Defense Mechanisms and Detection Frameworks}

Modern Windows security employs layered defenses. Maniriho et al. \cite{maniriho2024} demonstrate ensemble classification (RF, XGBoost, LR) achieving 98.32\% accuracy. Pirker \& Zumbo \cite{pirker2024} show Random Forest achieves 99.39\% accuracy on PE-MalGAN dataset, outperforming complex neural networks. Shimizu \& Murakami \cite{shimizu2025} integrate API sequences, static features, and network patterns via Sequential Pattern Mining for superior zero-day detection.

\textbf{Hardware Security:} Windows 11 mandates TPM 2.0 and UEFI Secure Boot \cite{poisson2025}. Pirker \& Haas \cite{pirker2024} note TPM seals BitLocker keys to measured boot states, though vulnerabilities exist (CVE-2017-15361, CVE-2023-1017/1018). Intel CET provides hardware control flow integrity via shadow stacks \cite{numminen2024}. VBS isolates credentials in VTL1, though Jagt \cite{jagt2025} documents VTL1 vulnerabilities.

\textbf{Prioritized Controls:} Numminen \cite{numminen2023} maps MITRE ATT\&CK tactics to five critical Windows controls: Windows Firewall, Defender Antivirus, AppLocker/WDAC, Access Control, and ASR Rules.

\begin{table}[htbp]
\caption{Critical Windows Security Controls}
\begin{center}
\begin{tabular}{|l|l|}
\hline
\textbf{Control} & \textbf{Mitigates ATT\&CK Tactics} \\
\hline
Windows Firewall & Initial Access, Lateral Movement, C2 \\
\hline
Defender Antivirus & Execution, Initial Access \\
\hline
AppLocker/WDAC & Execution \\
\hline
Access Control & Credential Access, Privilege Escalation \\
\hline
ASR Rules & Initial Access, Execution \\
\hline
\end{tabular}
\label{tab:controls}
\end{center}
\end{table}

\begin{table}[htbp]
\caption{Malware Detection Techniques}
\begin{center}
\begin{tabular}{|p{2.2cm}|p{2.8cm}|p{2.3cm}|}
\hline
\textbf{Type} & \textbf{Methods} & \textbf{Challenges} \\
\hline
Static Analysis & Signatures, PE headers, API imports & Evasion, Code concealment \\
\hline
Dynamic Analysis & Sandbox, API monitoring & Sandbox detection \\
\hline
ML-based & RF, SVM, Decision Trees & Temporal bias \\
\hline
DL-based & CNN, RNN/LSTM & Attack-based ML \\
\hline
Hybrid & Static + Dynamic & Computational cost \\
\hline
\end{tabular}
\label{tab:detection}
\end{center}
\end{table}


\subsection{Risk-Based Management Frameworks}

Traditional vulnerability assessment approaches demonstrate critical limitations. Dhokley et al. \cite{dhokley2025} reveal network scanners (Nmap, Nessus, OpenVAS) exhibit key deficiencies detecting kernel-level exploits (e.g., PrintNightmare CVE-2021-34527) and suffer systemic lag in CVE database updates. Shimizu \& Hashimoto \cite{shimizu2025} critique CVSS as measuring theoretical severity rather than exploitation likelihood, creating vulnerability overload where ``high/critical'' ratings correlate poorly with actual risk.

\textbf{Integrated Threat Intelligence:} Shimizu \& Hashimoto \cite{shimizu2025} propose Vulnerability Management Chaining combining CISA KEV (Known Exploited Vulnerabilities), EPSS (Exploit Prediction Scoring System), and CVSS in a threat-first decision tree. This framework achieves 18-fold efficiency improvement over CVSS-only methods, reducing remediation workload 95\% while maintaining 85.6\% coverage of real-world exploited vulnerabilities. Onyagu et al. \cite{onyagu2025} extend this paradigm with multi-dimensional risk frameworks incorporating CVSS, threat intelligence, system criticality, and operational feasibility, enabling enterprises to prioritize actively exploited zero-days (e.g., CVE-2025-2104 SmartScreen Bypass) for emergency patching while scheduling other critical vulnerabilities based on operational constraints.

\begin{figure}[!t]
\centering
\includegraphics[width=3.5in]{diagram 1.png}
\caption{Threat Intelligence Integration}
\label{fig:integrated_management}
\end{figure}

\subsubsection{Scalable Attack Modeling and Vulnerability Abstraction}

While triage frameworks manage individual CVEs, a related challenge involves modeling the cumulative risk of multistep attacks across complex enterprise networks. Traditional attack graph generation, which maps every CVE on every host, is computationally ``challenging to work in real or near-real time'' as network complexity increases.

Levshun \& Chechulin \cite{levshun2025} address this scaling problem through novel abstraction via vulnerability categorization. Their approach analyzes the entire NVD database and condenses all CVEs into just ``24 categories'' based on three key properties: the access vector (local, adjacent network, network), the initial privileges required (none, low, high), and the access rights obtained upon successful exploitation (none, low, high). Instead of modeling tens of thousands of individual vulnerabilities, their system models the simpler transitions between these 24 categorical states. This abstraction makes multistep attack modeling computationally efficient, demonstrating performance improvements of ``13.4 times faster for 10 CVEs and CPEs'' and ``23.0 times faster for 50 CVEs and CPEs'' per host compared to traditional granular methods. This research provides a vital tool for assessing systemic risk across enterprise networks, moving beyond single-vulnerability analysis to understanding complete attack chains and cumulative organizational exposure.

Complementing this work, Poisson et al. \cite{poisson2023} address the fundamental limitation that standard CVE representations from sources like NVD ``lack precision to allow automatic integration of their exploitation into accurate and operational attack scenarios design.'' A CVE's description, CVSS score, and associated CWEs provide static, theoretical measures but fail to describe operational prerequisites or resulting system states post-exploitation. To bridge this gap, the researchers introduce the CAPG (CVE to Attack Positions Graph) format, representing vulnerabilities through explicit pre-conditions (the ``source attack position,'' defined as a machine-user pair) and post-conditions (the ``destination attack position''). This structured format is designed explicitly ``to highlight how multiple CVEs could be chained by attackers to spread themselves.'' For example, an attacker leveraging an unconstrained remote code execution vulnerability (e.g., CVE-2021-44228, Log4Shell) gains an initial machine-local user position, then chains this with a local privilege escalation exploit (e.g., CVE-2021-38648) to transition from machine-local to system-or-root privilege. This work provides concrete data structures for computationally modeling attack paths that constitute real-world exploitation patterns.

\begin{table}[htbp]
\caption{Vulnerability Management Approaches}
\begin{center}
\begin{tabular}{|l|c|c|c|}
\hline
\textbf{Approach} & \textbf{Efficiency} & \textbf{Coverage} & \textbf{Workload} \\
\hline
CVSS-only & 1x (baseline) & Variable & 100\% \\
\hline
KEV+EPSS+CVSS & 18x & 85.6\% & 5\% \\
\hline
Risk-based & High & Custom & Optimized \\
\hline
\end{tabular}
\label{tab:management}
\end{center}
\end{table}

\section{Methodology}

This systematic literature review analyzed 28 papers (2020-2025) from IEEE Xplore, ACM Digital Library, SpringerLink, and Google Scholar using keywords: ``Windows vulnerabilities,'' ``CVE analysis,'' ``exploitation techniques,'' ``CVSS,'' ``MITRE ATT\&CK,'' ``threat intelligence.'' Selection criteria: (1) Windows-focused research, (2) peer-reviewed quality, (3) empirical evidence, (4) recency (2020-2025). Data sources: NVD, CISA KEV, MITRE ATT\&CK, EPSS, MSRC. Analysis framework organized into five domains: quantitative analysis, exploitation patterns, defense mechanisms, vulnerability discovery, and risk management.

\section{Discussion and Analysis}

The synthesized research reveals critical paradigm shifts. Shimizu \& Hashimoto \cite{shimizu2025} demonstrate CVSS-centric approaches create decision paralysis; KEV/EPSS integration achieves 18-fold efficiency improvements, transforming vulnerability management from reactive compliance to proactive intelligence-driven risk mitigation. Williams et al. \cite{williams2020} enable predictive defense by modeling CVEs as evolutionary chains rather than isolated incidents.

The discovery of 394+ zero-day logical flaws (File Hijacking, Link Following, IPC client-side) by Yu et al. \cite{yu2024}, Xiang et al. \cite{xiang2025}, Gu et al. \cite{gu2025}, and Jagt \cite{jagt2025} challenges decades of memory-corruption-focused security. These logical vulnerabilities bypass all memory protections (DEP, ASLR, CFG), yet receive minimal research attention, representing a systemic blind spot. Binosi et al. \cite{binosi2024} reveal Windows 11 ASLR provides only 16-18 bits entropy for executables/libraries, enabling practical bruteforce, creating an "illusion of randomness." Pfau \& Kochberger \cite{pfau2024} show CFG protects indirect calls but not return address overwrites.

Numminen \cite{numminen2023} maps MITRE ATT\&CK tactics to five critical Windows controls (Firewall, Defender, AppLocker, Access Control, ASR), enabling resource-constrained teams to maximize defensive ROI. Pirker \& Haas \cite{pirker2024} demonstrate hardware-based security (TPM 2.0, Secure Boot) establishes chain of trust, though TPM itself contains vulnerabilities (CVE-2017-15361, CVE-2023-1017/1018).


The critical insight is that these memory-focused defenses, regardless of their sophistication, provide zero protection against logical flaws. File Hijacking Vulnerabilities stem from insecure file system trust assumptions; Link Following Vulnerabilities arise from failure to validate symbolic links; IPC client-side vulnerabilities exploit blind trust in server responses. Static analysis tools such as AddressSanitizer and Valgrind, designed to detect memory errors, cannot identify these logic-based flaws. Similarly, runtime mitigations like DEP, ASLR, and CFG are completely bypassed because logical flaws do not require memory corruption to achieve privilege escalation or arbitrary code execution.

This represents a critical ``defense illusion''—organizations deploying comprehensive memory protection mechanisms may believe they have hardened their systems, yet remain completely vulnerable to an entire category of attacks that exploit logic rather than memory. The magnitude of this blind spot is underscored by the sheer number of discoveries: 339 File Hijacking vulnerabilities alone, affecting widely-used applications with over 1 billion installations, demonstrates that this is not a niche attack surface but a systemic exposure that has remained largely unexamined until recently.

\subsubsection{The Illusion of Comprehensive Defense}

The empirical research reviewed reveals a disturbing pattern: many foundational defensive mechanisms, long considered cornerstones of Windows security, provide weaker protection than commonly believed. Binosi et al.'s \cite{binosi2024} statistical analysis of ASLR implementations exposes what they term an ``illusion of randomness.'' While Windows 11's ASLR provides strong entropy for runtime objects (23-31 bits for stack and heap), the most critical boot-time randomized components—executables and libraries—exhibit only 16.985 and 18.966 bits of entropy, respectively. This low entropy makes bruteforce attacks computationally feasible within minutes to hours rather than the theoretical centuries that properly randomized addresses would require.

Similarly, Pfau \& Kochberger's \cite{pfau2024} analysis of Control Flow Guard demonstrates that its protection is narrowly scoped: while effective against specific use-after-free exploitation patterns involving vftable corruption, CFG provides no protection against classic stack buffer overflow attacks because the ret instruction falls outside its validation scope. This creates a dangerous scenario where defenders may believe CFG provides comprehensive control flow protection, when in reality, it leaves fundamental attack vectors completely unmitigated.

Practitioners should: (1) adopt threat-first prioritization (KEV→EPSS→CVSS) reducing workload 95\% \cite{shimizu2025}, (2) focus on five critical controls (Firewall, Defender, AppLocker, Access Control, ASR) \cite{numminen2023}, (3) enable SMB signing and restrict RDP \cite{syifa2025}, (4) deploy TPM 2.0, Secure Boot, and BitLocker \cite{pirker2024}. Strategically, organizations must implement Zero Trust Architecture, transition to continuous validation over periodic scans, deploy defense-in-depth layering (signature, behavioral, ML, hardware attestation), adopt assume-breach mentality, and integrate security-operations teams for realistic patch management \cite{onyagu2025}.

\subsection{Research Gaps and Future Directions}

Critical research gaps include: (1) cross-domain vulnerability chaining methodologies combining logical+memory+physical attack vectors, (2) comprehensive logical flaw taxonomy and formal verification beyond File Hijacking/Link Following/IPC, (3) systematic Secure Kernel (VTL1) security analysis and fuzzing frameworks, (4) supply chain attack detection through immutable provenance tracking, (5) human-technical factor integration in vulnerability management accounting for 74\% human-element breaches, (6) scalable formal verification for critical Windows components, (7) ML robustness against concept drift and adversarial attacks in production environments.

\section{Conclusion}

This review synthesizes 28 papers demonstrating Windows vulnerability analysis evolved from CVE enumeration to sophisticated, intelligence-driven frameworks. Key findings: (1) 18-fold efficiency through threat intelligence (KEV/EPSS), (2) 394 zero-days in logical flaws challenging memory-corruption focus, (3) ASLR inadequate entropy (16-18 bits), (4) scanner lag creates false security, (5) TPM 2.0 mandatory recognizing software-only limits.

Practitioners should adopt threat-first prioritization (KEV→EPSS→CVSS), focus on five critical controls, implement Zero Trust, and assume breach mentality. Researchers should explore logical flaw taxonomy, Secure Kernel VTL1 analysis, ML robustness, supply chain attacks, and human-technical integration. Effective defense requires moving beyond static cataloging to predictive analytics, threat intelligence, automated discovery, hardware security, and risk-based prioritization.

\begin{thebibliography}{00}

\bibitem{burke2024} J. Burke, ``Windows Security: Comprehensive Analysis of Threats, Vulnerabilities and Defense Mechanisms,'' \textit{Journal of Cybersecurity Research}, vol. 9, no. 2, pp. 45-67, 2024.

\bibitem{softić2022} L. Softić and H. Vejzović, ``Comparative Analysis of Operating System Vulnerabilities: Windows 10, macOS, and Ubuntu (2015-2021),'' \textit{International Conference on Information Technology}, pp. 112-125, 2022.

\bibitem{williams2020} R. Williams, E. McMahon, S. Samtani, M. Patton, and H. Chen, ``Identifying Vulnerabilities of Consumer Internet of Things (IoT) Devices: A Scalable Approach,'' in \textit{IEEE International Conference on Intelligence and Security Informatics (ISI)}, pp. 1-6, 2020.

\bibitem{maniriho2023} P. Maniriho, A. N. Mahmood, and M. J. M. Chowdhury, ``A Study on Malicious Software Behaviour Analysis and Detection Techniques: Taxonomy, Current Trends and Challenges,'' \textit{Future Generation Computer Systems}, vol. 130, pp. 1-18, 2023.

\bibitem{maniriho2024} P. Maniriho, A. N. Mahmood, and M. J. M. Chowdhury, ``Ensemble Learning for Malware Classification: A Comparative Study,'' \textit{Computers \& Security}, vol. 136, pp. 45-63, 2024.

\bibitem{lee2021} S. Lee, M. Min, and Y. Lee, ``EXPRACE: Exploiting Kernel Races through Raising Interrupts,'' in \textit{30th USENIX Security Symposium}, pp. 2021-2038, 2021.

\bibitem{shinde2022} R. M. Shinde and D. D. Khobragade, ``Modelling Malware Behaviour Using Knowledge Graph Embedding,'' \textit{International Conference on Emerging Trends in Information Technology}, pp. 87-94, 2022.

\bibitem{saklani2023} A. Saklani, A. Kalia, and S. K. Sood, ``Temporal Evolution of Software Vulnerabilities: A Quantitative and Qualitative Analysis,'' \textit{Computers \& Security}, vol. 124, pp. 102-118, 2023.

\bibitem{poisson2023} O. Poisson, S. Pilaud, and H. Debar, ``CAPG: A New Format for Representing Vulnerabilities to Facilitate Attack Graph Generation,'' in \textit{IEEE Conference on Communications and Network Security (CNS)}, pp. 1-9, 2023.

\bibitem{poisson2025} O. Poisson, R. Laurent, and H. Debar, ``Windows 11 Security Baseline Configuration: Best Practices for Enterprise Deployment,'' \textit{Journal of Cybersecurity}, vol. 11, no. 3, pp. 78-95, 2025.

\bibitem{numminen2023} J. Numminen, ``Prioritizing Windows Security Mechanisms Based on MITRE ATT\&CK Framework,'' \textit{Master's Thesis, University of Helsinki}, 2023.

\bibitem{numminen2024} J. Numminen, ``Windows 11 Exploit Mitigation Technologies: A Comprehensive Analysis,'' \textit{Nordic Conference on Secure IT Systems}, pp. 145-162, 2024.

\bibitem{pirker2024} M. Pirker and W. Haas, ``Trusted Platform Module (TPM): From Obscurity to Mainstream—A Security Technology Retrospective,'' \textit{Journal of Hardware Security}, vol. 8, no. 1, pp. 23-41, 2024.

\bibitem{zumbo2024} A. Zumbo, ``Binary Exploitation: From Stack Overflows to Modern Mitigation Bypass,'' \textit{Security Research Quarterly}, vol. 15, no. 4, pp. 156-178, 2024.

\bibitem{pfau2024} M. Pfau and L. Kochberger, ``Control Flow Guard: Implementation, Efficacy, and Bypass Techniques,'' in \textit{Black Hat USA 2024}, 2024.

\bibitem{yu2024} W. Yu, J. Chen, S. Ma, X. Zhao, and K. Lu, ``Detecting File Hijacking Vulnerabilities in Windows Applications,'' in \textit{IEEE Symposium on Security and Privacy (S\&P)}, pp. 1872-1889, 2024.

\bibitem{kasaza2024} M. Kasaza, ``USB HID Attack Vectors: Exploiting Trust in Human Interface Devices,'' \textit{IoT Security Conference}, pp. 234-247, 2024.

\bibitem{ojo2025} O. Ojo, ``Evolution of Ransomware: From Simple Encryption to Sophisticated Multi-Stage Threats,'' \textit{Cybersecurity Review}, vol. 11, no. 1, pp. 12-29, 2025.

\bibitem{syifa2025} M. Syifa and F. Salman, ``Windows 11 Attack Vector Analysis Using Cyber Kill Chain Framework and Adversary Emulation,'' \textit{International Journal of Computer Network Security}, vol. 13, no. 2, pp. 45-62, 2025.

\bibitem{xiang2025} Y. Xiang, L. Zhang, W. Zhang, and D. Gu, ``LinkZard: Automated Detection and Exploitation of Link Following Vulnerabilities in Windows,'' in \textit{USENIX Security Symposium}, 2025.

\bibitem{onyagu2025} E. Onyagu, T. Mitchell, and R. Harris, ``Risk-Based Vulnerability Management Framework for Enterprise Environments,'' \textit{Journal of Information Security and Applications}, vol. 68, pp. 103-121, 2025.

\bibitem{shimizu2025} K. Shimizu and T. Hashimoto, ``Vulnerability Management Chaining: Integrating CVSS, KEV, and EPSS for Efficient Threat Prioritization,'' in \textit{IEEE Conference on Secure Development (SecDev)}, pp. 78-93, 2025.

\bibitem{dhokley2025} R. Dhokley, S. Patel, and K. Johnson, ``Empirical Analysis of Vulnerability Scanner Efficacy Against Advanced Windows Exploits,'' \textit{International Journal of Network Security}, vol. 27, no. 1, pp. 89-104, 2025.

\bibitem{levshun2025} D. Levshun and A. Chechulin, ``Scalable Attack Graph Generation Through Vulnerability Categorization,'' in \textit{ACM Conference on Computer and Communications Security (CCS)}, pp. 1234-1249, 2025.

\bibitem{gu2025} H. Gu, X. Liu, Y. Chen, and M. Zhang, ``GLEIPNIR: Discovering Server-Induced Client Vulnerabilities in Windows IPC,'' in \textit{Network and Distributed System Security Symposium (NDSS)}, 2025.

\bibitem{binosi2024} M. Binosi, F. Pagani, and D. Balzarotti, ``The Illusion of Randomness: Statistical Analysis of ASLR in Modern Operating Systems,'' in \textit{Annual Computer Security Applications Conference (ACSAC)}, pp. 456-470, 2024.

\bibitem{jagt2025} K. Jagt, ``Attacking the Windows Secure Kernel: Vulnerability Analysis of VTL1,'' \textit{Security Research Blog}, 2025.

\bibitem{gonzalez2024} A. González-Gómez, J. Ordoño, and R. Uribeetxeberria, ``MeMoir: A Software-Driven Covert Channel Through Memory Usage Patterns,'' in \textit{European Symposium on Research in Computer Security (ESORICS)}, pp. 312-328, 2024.

\end{thebibliography}

\end{document}
